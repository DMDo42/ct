\def\pathToRoot{../../}\documentclass{article}

\usepackage{nag}
\usepackage[small,compact]{titlesec}
\usepackage[utf8]{inputenc}
\usepackage[T1]{fontenc}
\usepackage{lmodern}
\usepackage{color}
\usepackage{parskip}
\usepackage{needspace}
\usepackage{microtype}
\usepackage{mathtools}
\usepackage{xifthen}
\usepackage{xpatch}
\usepackage{enumitem}
\usepackage{mdwlist}
\usepackage{bussproofs}
\EnableBpAbbreviations
\usepackage{tabu}
\usepackage{amssymb}
\usepackage{amsmath}
\usepackage{amsthm}
%grober hack, der den groben hack von parskip bei den amsthm sachen korrigiert
\begingroup
    \makeatletter
       \@for\theoremstyle:=definition,remark,plain\do{%
            \expandafter\g@addto@macro\csname th@\theoremstyle\endcsname{%
                        \addtolength\thm@preskip\parskip
             }%
        }
\endgroup
\usepackage[UKenglish]{babel}
\usepackage{xparse}
\usepackage{adjustbox}
\usepackage{geometry}
\usepackage{booktabs}
\usepackage{multicol}
\usepackage{soul}
\usepackage{calc}
\usepackage{textcase}
\usepackage{stmaryrd}
\usepackage{marvosym}
\usepackage{wasysym}
\usepackage{pifont}
\newcommand{\cmark}{\ding{51}}
\newcommand{\xmark}{\ding{55}}
\usepackage{tikz}
\usetikzlibrary{trees, backgrounds, shapes, chains, decorations.text, decorations.pathreplacing, circuits.logic.IEC, patterns, matrix}
\usepackage{tikz-qtree}
\usepackage{tikzsymbols}
\usepackage{fancyvrb}
\usepackage{fancyhdr}
\usepackage{verbatim}
\usepackage[framemethod=tikz]{mdframed}
\usepackage{lastpage}
\usepackage{pgfpages}
\usepackage{csquotes}
\usepackage{longtable}
\usepackage{ragged2e}
%\usepackage{stackengine}
\usepackage{censor}
\usepackage{expl3}
\usepackage{multirow}
\usepackage{hyperref}
\usepackage{environ}


% Package for Cateogry diagrams:

\usepackage{tikz-cd}




\renewcommand{\labelenumi}{(\alph{enumi})}
\renewcommand{\labelenumii}{(\roman{enumii})}

\input{\pathToRoot headers/definitions}



\tikzset{
    normal/.style={draw, semithick},
    n/.style={style=normal, circle, inner sep=1mm, minimum size=8mm},
    l/.style={style=normal, rounded corners=1mm, inner sep=1mm, minimum size=6mm},
    e/.style={style=normal, shorten >=1mm, shorten <=1mm, ->, >=stealth},
    syntax/.style={style=normal, ellipse, minimum height=6mm, minimum width=8mm}, % nodes in syntax trees
    inner/.style={style=normal, minimum size=4mm}, % inner leaves or root in normal trees
    leaf/.style={style=normal, circle, minimum size=4mm}, % leaves in normal trees
    te/.style={style=normal}, % edges in a tree
    be/.style={style=e, dashed} % binding edge
}

\newcommand{\syntaxtree}[1]{ % DEPRECATED - use tikzsyntaxtree
    \begin{tikzpicture}[baseline=(current bounding box.north)]
        \tikzset{grow=down}
        \tikzset{every node/.style={syntax}}
        \tikzset{edge from parent/.style=
            {te,
                edge from parent path={(\tikzparentnode) -- (\tikzchildnode)}}}
        \Tree #1
    \end{tikzpicture}
}

\newenvironment{tikzsyntaxtree}[1][]{
    \begin{tikzpicture}[baseline=(current bounding box.north), #1]
    \tikzset{grow=down}
    \tikzset{every tree node/.style={syntax}}
    \tikzset{edge from parent/.style={te, edge from parent path={(\tikzparentnode) -- (\tikzchildnode)}}}
}{
    \end{tikzpicture}
}


\newcommand{\DisplayScaledProof}{\maxsizebox{\linewidth}{!}{\DisplayProof}}
\newcommand{\DisplayTopProof}{\adjustbox{valign=t}{\DisplayProof}}
\newcommand{\DisplayScaledTopProof}{\adjustbox{valign=t}{\maxsizebox{\linewidth}{!}{\DisplayProof}}}


\newcolumntype{P}[1]{>{\RaggedRight\hspace{0pt}}p{#1}}

\newenvironment{prooftable}
{
    \begin{longtable}{>{\footnotesize}p{0.33\textwidth}>{\footnotesize}p{0.33\textwidth}|>{\footnotesize}P{0.15\textwidth}}
    \normalsize Textbeweis & \normalsize Erklärungen & \normalsize Schlussregel\\\hline
    \endhead
}
{
    \end{longtable}
}


\theoremstyle{definition}
\newtheorem*{definition*}{Definition} % Definition ohne Nummer
\newtheorem*{inferenceRule*}{Schlussregel}

\usepackage{titling}
\geometry{a4paper,left=2cm,right=2cm,top=2cm,bottom=3cm}


\newcommand{\licenseccjuliachristian}{\def\islicenseccjuliachristian{}}
\newcommand{\suppresslicense}{\def\issuppresslicense{}}


\AtBeginDocument{
    \pagestyle{fancy}
    \renewcommand{\headrulewidth}{0pt}
    \renewcommand{\footrulewidth}{1pt}
    \fancyhead{}
    \fancyfoot[C]{\thepage~/~\pageref{LastPage}}
    \fancyfoot[R]{\footnotesize exercise sheet from \\ \theauthor}

}


\newcommand{\pgbreakhere}{\Needspace*{4\baselineskip}}
\newcommand{\pgbreakHere}{\Needspace*{10\baselineskip}}
\newcommand{\pgbreakHERE}{\Needspace*{15\baselineskip}}

\newcommand{\raisedrule}[2][0em]{\leavevmode\leaders\hbox{\rule[#1]{1pt}{#2}}\hfill\kern0pt}

% inspired by http://tex.stackexchange.com/questions/242294/suppress-parskip-only-after-a-specific-paragraph
\makeatletter
\newlength\noparskip@parskip % used to store a backup of the parskip value
\newboolean{noparskip@triggered} % flag to indicate that noparskip was run in the current paragraph
\setboolean{noparskip@triggered}{false}
\newboolean{noparskip@active} % flag to indicate that parskip should be restored after this paragraph
\setboolean{noparskip@active}{false}
\let\noparskip@par\par % store a backup of the \par command
\@setpar{% redefine \par with the means of ltpar.dtx to stay compatible to enumerate and itemize
    \ifhmode% since we're counting occurrences of \par, \par\par would be a problem, so check that we are actually ending a paragraph
        \ifthenelse{\boolean{noparskip@active}}{%
            \setlength\parskip\noparskip@parskip% restore parskip
            \setboolean{noparskip@active}{false}% remember not the restore parskip again
        }{}%
        \ifthenelse{\boolean{noparskip@triggered}}{%
            \ifthenelse{\boolean{noparskip@active}}{}{
                % we are triggering noparskip and not currently in a noparskip already
                \setlength\noparskip@parskip\parskip % copy the current parskip into the backup variable
            }%
            \setboolean{noparskip@triggered}{false}% paragraph is ending, so noparskip is no longer triggered
            \setlength\parskip{0pt}% no parskip when the next paragraph begins
            \setboolean{noparskip@active}{true}% parskip must be restored by the next par
        }{}%
    \fi%
    \noparskip@par% run the original par command
}
\def\noparskip@backout{%
    \ifthenelse{\boolean{noparskip@active}}{%
        % a list is beginning and parskip is currently set to zero, wich would mess up the list
        \setlength\parskip{\noparskip@parskip}% restore parskip before the list begins
        \setboolean{noparskip@active}{false}%
    }{}%
    \setboolean{noparskip@triggered}{false}% there's no sense in keeping noparskip triggered throughout a list
}
\xpretocmd\begin{%
    \ifstrequal{#1}{enumerate}{\noparskip@backout}{}%
    \ifstrequal{#1}{itemize}{\noparskip@backout}{}%
    \ifstrequal{#1}{list}{\noparskip@backout}{}%
    \ifstrequal{#1}{proof}{\noparskip@backout}{}%
}{}{}
\def\noparskip{%
    \leavevmode% ensure that we are within a paragraph
    \setboolean{noparskip@triggered}{true}% trigger noparskip
}
\makeatother

\newcommand{\noparskipworkaround}{} % DEPRECATED and no longer needed


\newcommand{\head}[1]{
    {
        \setlength{\parskip}{0pt}
        \hrule height 1pt
        \vspace{.2cm}
        Saarland University \hfill Category Theory Seminar 2017\par
        Programming Systems Lab \hfill \small\url{https://courses.ps.uni-saarland.de/ct_ss17/}\par
        \tiny\raisedrule[0mm]{1pt}
        \vspace{2ex}
        \begin{center}
            \Large
            \textbf{#1}\par
            \raisedrule[2mm]{1pt}
        \end{center}
        \vspace{3ex}
    }
}

\newenvironment{leftframedparagraph}{\begin{mdframed}[hidealllines = true, leftline = true, innerleftmargin = 2ex, innerrightmargin = 0pt,
innertopmargin = 0pt, innerbottommargin = 2pt, skipabove=2ex, skipbelow=1ex, outerlinewidth = 0ex, innerlinewidth = 0.5ex]}{\end{mdframed}}
\newenvironment{leftframed}{\begin{mdframed}[hidealllines = true, leftline = true, innerleftmargin = 2ex, innerrightmargin = 0pt,
innertopmargin = 0pt, innerbottommargin = 0pt, skipabove=2ex, skipbelow=1ex, outerlinewidth = 0ex, innerlinewidth = 0.5ex]}{\end{mdframed}}

%%% Local Variables:
%%% mode: latex
%%% TeX-master: t
%%% End:


\newcommand{\uebunghead}[3][Exercise sheet:]{\def\sheetid{#2}\head{#1 #2\ifthenelse{\isundefined{\issolution}}{}{ \ifthenelse{\isundefined{\ismarking}}{(Possible solutions)}{(Marking)}} \\ #3}}

\licenseccjuliachristian


\newcommand{\amountofpoints}[1]{\ifstrequal{#1}{1}{1~Punkt}{#1~Punkte}}


% marking implies solution
\ifthenelse{\isundefined{\ismarking}}{}{\def\issolution{}}


%%%Environments
\newcounter{ExamExerciseCounter} % will only be used in exams, but must be defined here so ExerciseCounter can be reset when ExamExericise counts
\setcounter{ExamExerciseCounter}{0}
\newcounter{ExerciseCounter}[ExamExerciseCounter]
\setcounter{ExerciseCounter}{0}

\newcommand{\ExerciseNumber}{\sheetid.\arabic{ExerciseCounter}}
\renewcommand{\theExerciseCounter}{\ExerciseNumber}

\newcommand{\ExercisePointHook}[1]{}

%Aufgaben-Umgebung
\NewDocumentEnvironment{exercise}{od<>}{
    \refstepcounter{ExerciseCounter}
    \pgbreakhere
    \vspace{1ex}\textbf{Exercise\ \ExerciseNumber}%
    \IfNoValueF{#1}{ \emph{(#1)}}%
    \IfNoValueF{#2}{\hfill(\amountofpoints{#2})}%
    \IfNoValueF{#2}{\ExercisePointHook{#2}}%
    \noparskip\par\nopagebreak
}{
    \par
    \vspace{2ex}
}



%Loesungs-Umgebung
\newenvironment{answer}
{
    \ifthenelse{\isundefined{\issolution}}
    {
        \comment
    }{
        \vspace{1ex}\textsl{Lösungsvorschlag \ExerciseNumber}\noparskip\par\nopagebreak
    }
}{
    \ifthenelse{\isundefined{\issolution}}
    {
    }{
        \vspace{1ex}
        \hspace*{\fill}
    }
}

\newenvironment{marking}
{%
    \ifthenelse{\isundefined{\ismarking}}%
    {%
        \comment%
    }{%
        \color{red}
    }%
}{%
    \ifthenelse{\isundefined{\ismarking}}%
    {%
    }{%
    }%
}

\newenvironment{example}{\begin{leftframedparagraph}\paragraph{Example:}}{\end{leftframedparagraph}}
\newenvironment{hint}{\paragraph{Hint:}}{}
\newenvironment{caution}{\paragraph{Caution:}}{}
\newenvironment{definition}[1]{\begin{leftframedparagraph}\paragraph{Definition (#1):}}{\end{leftframedparagraph}}


\DeclareMathOperator{\A}{\mathscr A}
\DeclareMathOperator{\B}{\mathscr B}
\DeclareMathOperator{\M}{\mathscr M}

\begin{document}

% Use Basis x or Talk x, where x is the number of the session
\uebunghead{Basis 2}{Maps between Categories -- Functors}

\author{Mostafa Ahmed Abdelfattah Abouhamra, Felix Rech, Dominik Wagner}

\begin{hint}
  Read Chapter 1.2.
\end{hint}

\begin{exercise}
Show that functors preserve isomorphism. That is, prove that if $F : \cat{A} \to \cat{B}$ is a functor and $A, A' \in \cat{A}$ with $A \iso A'$, then $F(A) \iso F(A')$.
\end{exercise}

\begin{exercise}
  Check that there is a category $\Cat$ with categories as objects and functors as arrows.
\end{exercise}

\begin{exercise}
  Find an example of a functor $F : \cat{A} \to \cat{B}$ such that $F$ is faithful but there exist distinct maps $f_1$ and $f_2$ in $\cat{A}$ with $F(f_1) = F(f_2)$.
\end{exercise}

\begin{exercise}
  Find interesting examples for a functor that is both full and faithful, one that is full but not faithful, one that is faithful but not full, and one that is neither.
\end{exercise}

\begin{exercise}
  \begin{enumerate}
    \item What are the subcategories of an ordered set? Which are full?
    \item What are the subcategories of a group? (Careful!)
      Which are full?
  \end{enumerate}
\end{exercise}

\begin{exercise}
  \begin{enumerate}
    \item Define an endofunctor on $\Set$ that maps every set to its powerset.
      Check that it satisfies the functor laws.
      Is it full/faithful?
    \item Define a contravariant endofunctor on $\Set$, i.e.\ a functor from $\Set^\op$ to $\Set$, that maps every set to its powerset.
      Check that it satisfies the functor laws.
      Is it full/faithful?
  \end{enumerate}
\end{exercise}

\begin{exercise}
  Let $A$ and $B$ be posets, and $\A$ and $\B$ be the corresponding categories. Consider a contravariant functor $F:\A\rightarrow \B$. Describe properties of the corresponding function $f:A\rightarrow B$.
\end{exercise}

\begin{exercise}
  Determine which of the following holds, i.e.\ either provide an isomorphism in $\mathbf{CAT}$ or argue why there is none:

  \begin{enumerate}
  \item $\mathbf {Rel}^{\op}\cong \mathbf {Rel}$;
  \item $\mathbf {Set}^{\op}\cong \mathbf {Set}$;
  \item for an arbitrary monoid interpreted as a category $\M$ with one object:  $\M^{\op}\cong \M$; \textit{(Not entirely trivial 1)}
  \item for $\mathbf{CAT_i}$, the subcategory of $\mathbf{CAT}$ consisting of all categories and functors that are isomorphisms: $\mathbf{CAT_i}^{op}\cong\mathbf{CAT_i}$.
  \end{enumerate}

\end{exercise}

\begin{definition}{Free Monoid}
  For every set $A$, we define the \emph{free monoid} $A^* \coloneqq \lbrace \text{words over $A$} \rbrace$.
  We define the binary operation by $w \cdot w' \coloneqq ww'$ as the concatenation of words and use the empty word $\varepsilon$ as neutral element.
\end{definition}

\begin{exercise}
  Show that for every set $A$, the free monoid $A^*$ satisfies the following \emph{universal mapping property}:
  There is a function $i : A \to A^*$ and given any monoid $M$ and any function $j : A \to M$, there is a \emph{unique} monoid homomorphism $f : A^* \to M$ such that $f \circ i = j$.
\end{exercise}

\begin{exercise}
  Denote by $\Mon$ the category of monoids with monoid morphisms as arrows.
  Define a functor $F : \Set \to \Mon$ that maps every set $A$ to the free monoid $A^*$.
  Is $F$ full/faithful?
\end{exercise}

\begin{exercise}
  \begin{enumerate}
    \item A directed graph consists of a set of vertices $V$, a set of edges $E$ and two functions $s, t : E \to V$ that assign a source and target vertex to each edge.
      Define a category $\cat{A}$ such that functors from $\cat{A}$ to $\Set$ correspond to directed graphs.
    \item Define a category $\cat{B}$ such that the functors from $\cat{B}$ to $\Set$ correspond to injections between an arbitrary pair of sets.
  \end{enumerate}
\end{exercise}


\begin{definition}{Coslice Category}
  Let $\A$ be a category and $A\in\ob(\A)$ an object. The \emph{coslice category} $A/\A$ of $\A$ under $A$ consists of
  \begin{itemize}
  \item the arrows from $A$ (i.e. the domain of which is $A$) as objects
  \item and for $f\in \A(A,B)$, $g\in\A(A,B')$ (which are objects in $A/\A$) the arrows between $f$ and $g$ in $A/\A$ are arrows $h$ such that $h\of f=g$.
  \end{itemize}
\end{definition}

\begin{definition}{$\mathbf{Sets_*}$}
  The category $\mathbf{Sets_*}$ of \emph{pointed sets} consists of
  \begin{itemize}
  \item pairs $(A,a)$ such that $A$ is a set and $a\in A$ as objects and
  \item arrows between objects $(A,a)$ and $(B,b)$ are functions $f\from A\to B$ such that $f(a)=b$.
  \end{itemize}
\end{definition}

\begin{exercise}
Show that $\mathbf{Sets_*\cong 1/\mathbf{Sets}}$ where $1$ is an arbitrary set with exactly one element.  
\end{exercise}

\begin{definition}{Slice Category}
  Let $\A$ be a category and $A\in\ob(\A)$ an object. The \emph{slice category} $\A/A$ of $\A$ over $A$ consists of
  \begin{itemize}
  \item the arrows to $A$ (i.e. the codomain of which is $A$) as objects
  \item and for $f\in \A(B,A)$, $g\in\A(B',A)$ (which are objects in $A/\A$) the arrows between $f$ and $g$ in $A/\A$ are arrows $h$ such that $g\of h=f$.
  \end{itemize}
\end{definition}

\begin{exercise}[\textit{Not entirely trivial 2}]
  Let $\A$ be a category. Construct a functor $F\from \A\to\mathbf{Cat}$ such that $F(A)=\A/A$ and show that this is indeed a functor. Is this also an isomorphism?

  \begin{hint}
    For arrows $f\in\A(A,B)$ consider functors $F_f\from\A/A\to\A/B$ that act on objects $g\in \A(C,A)$ by $F_f(g)=f\of g$. 
  \end{hint}
\end{exercise}


\end{document}


% \begin{exercise}[\textit{Challenge}]
% Let $\A$ be a category. Consider again the functor $F_B\from \A\to\mathbf{Hom}(-,B)$ that is defined for a fixed $B\in\ob(A)$ by
% \begin{align*}
%   F_B(A)&=\A(A,B)\\
%   F_B(f)&=F_{B,f}
% \end{align*}
% for all $A,A'\in\ob(\A)$, $f\in \A(A,A')$ and
% \begin{align*}
%   F_{B,f}(g)=g\of f
% \end{align*}
% for all $g\in\A(A,B)$.

% Fill in the details of the construction of the functor $F\from \A^{\op} \to \mathbf{CAT}$ that satisfies $F(B)=\mathbf{Hom}(-,B)$, i.e. define its behaviour on arrows and verify that this is indeed a functor.

% \begin{hint}
%   Consider the functors $F_f\from\mathbf{Hom}(-,B)\to\mathbf{Hom}(-,B')$ for $f\in\A(B',B)$ given by
%   \begin{align*}
%     F_f(\A(A,B))&=\{f\of g\mid g\in \A(A,B)\}\\
%     F_f(g)&=F_{f,g}
%   \end{align*}
%   where $g\in\mathbf{Hom}(-,B)(\A(A,B),\A(A',B))$ and the function $F_{f,g}$ is defined by
%   \begin{align*}
%     F_{f,g}(h)= g(f\of h).
%   \end{align*}
%   for each arrow $h\from\A(A,B')$.
% \end{hint}

% % Is this functor also an isomorphism

% % given by
% %   \begin{align*}
% %     F(B)&=\mathbf{Hom}(-,B)\\
% %     F(f)&=F_f
% %   \end{align*}
% % where $A\in\ob(\A)$ and $f\in\A(B',B)$, each functor $F_f\from \mathbf{Hom}(-,B)\to\mathbf{Hom}(-,B')$ is given by
% % \begin{align*}
% %   F_f(\A(A,B))&=\{f\of g\mid g\in \A(A,B)\}\\
% %   F_f(g)&=F_{f,g}
% % \end{align*}
% % where $g\in\mathbf{Hom}(-,B)(\A(A,B),\A(A',B))$ and
% % \begin{align*}
% %   F_{f,g}(h)= g(f\of h).
% % \end{align*}
% % for each function $h\from\A(A,B)\to\A(A',B)$
% % Check that this is indeed a functor, i.e. satisfies the axioms, and determine whether it is also an isomorphism.
% \end{exercise}


% %\section{Diagrams}
% %
% %\begin{hint}
% %  This is how you can draw diagrams:
% %  \[
% %    \begin{tikzcd}
% %                                    & A \arrow{dr}{g}    & \\
% %      B \arrow{ur}{f} \arrow{rr}{h} &                    & C
% %    \end{tikzcd}
% %  \]
% %\end{hint}

% \section{Categories - Definition and Basics}

% \begin {exercise}
% Show there can be at most one inverse for a morphism $f \from A \to B$.
% \end{exercise}

% \begin {definition}{Rel}
% The objects of the category \textbf{Rel} are sets. The morphisms $f \from A \to B$ are subsets $f \sub A \times B$.
% The identity morphism on set $A$ is the equality relation $\{\round{a,a} \such a \in A\} := 1_A$.
% Composition of two morphisms $f \sub A \times B$, $g \sub B \times C$ is defined as
% \[ g \of f = \{ \round{a, c} \in A \times C \such \exists b. \round{a, b} \in f \land \round{b, c} \in g\} \]
% \end{definition}

% \begin {exercise}
% Show that \textbf{Rel} is a category.
% \end{exercise}


% \begin{definition}{Pos} The objects of  \textbf{Pos} are partially ordered sets
% (recall that a poset is a set $A$ equipped with a reflexive, transitive and antisymmetric binary relation $\leq^A$). The morphisms $m \from A \to B $ are monotone functions: $a \leq^A a' \implies m(a) \leq^A m(a')$
% \end{definition}

% \begin {exercise}
% Show that \textbf{Pos} is a category.
% \end{exercise}

% % \begin{exercise}
% % For a fixed set X with Powerset $\mathcal{P}(X)$, does $\mathcal{P}(X) \cong \mathcal{P}(X)^{op}$ hold?
% % \end{exercise}

% \begin{definition}{Pointed Category} If an object is both \emph{initial} and \emph{terminal}, it is called a \emph{zero object}. A \emph{pointed category} is one with a zero object.
% \end{definition}

% \begin{exercise}
%   \begin{enumerate}
%   \item Show that \textbf{Rel} is a pointed category.
%   \item Show that the category \textbf{Grp} of groups has both an initial and a final object, and that these are the same.
%   \item Show that the category \textbf{Ring} of unital rings has both an initial and a final object, and that these are \emph{not} the same.
%   \end{enumerate}
% \end{exercise}

% \begin{exercise}
% Show that any isomorphism is both monic and epic.
% \end{exercise}

% \newpage

% \begin{exercise}
%   Let $A, B, C$ be objects in a category and let $f \from A \to B$ and $g \from B \to C$ be morphisms such that the following diagram commutes:
%   \[
%     \begin{tikzcd}
%       A \arrow{r}{f} \arrow{dr}{h} & B \arrow{d}{g} & \\
%       & C &
%     \end{tikzcd}
%   \]
%   Show the following facts:
%   \begin{enumerate}
%   \item If both $f$ and $g$ are isomorphisms, then $h$ is also an isomorphism.
%   \item If $h$ is monic, then $f$ is monic.
%   \item If $h$ is epic, then $g$ is epic.
%   \end{enumerate}
% \end{exercise}

% \begin{exercise}
%   Show the following for a morphism $f \from A \to B$ in the category \textbf{Set}:
%   \begin{enumerate}
%   \item If $f$ is monic, then it is injective.
%   \item If $f$ is epic, then it is surjective. \textit{(Challenge 1)}
%   \end{enumerate}
% \end{exercise}

% \begin{exercise}[Challenge 2]
%   Show that in the category \textbf{Ring}, there are morphisms which are epic but not surjective.
%   \begin{hint}
%     Consider the inclusion morphism $\mathbb{Z} \to \mathbb{Q}$, and use the properties of ring homomorphisms.
%   \end{hint}
% \end{exercise}

\end{document}

% %%% Local Variables:
% %%% mode: latex
% %%% TeX-master: t
% %%% End:
