%Mathe
\newcommand{\mi}[1]{\mathit{#1}}
\newcommand{\with}{\;\middle|\;}
\newcommand{\round}[1]{\left(#1\right)}
\newcommand{\roundd}[1]{(#1)}
\newcommand{\rounddd}[1]{\big(#1\big)}
\newcommand{\roundddd}[1]{\Big(#1\Big)}
\newcommand{\rounddddd}[1]{\bigg(#1\bigg)}
\newcommand{\roundddddd}[1]{\Bigg(#1\Bigg)}
\newcommand{\brackets}[1]{\left[#1\right]}
\newcommand{\curly}[1]{\left\lbrace#1\right\rbrace}
\newcommand{\angles}[1]{\left\langle#1\right\rangle}
\newcommand{\semantics}[2][]{\left\llbracket#2\ifthenelse{\isempty{#1}}{\right\rrbracket}{\right\rrbracket_{#1}}}
\newcommand{\abs}[1]{\left|#1\right|}
\newcommand{\set}[1]{\curly{#1}}
\newcommand{\where}{\,\middle|\,}
\newcommand{\NN}{\mathbb{N}}
\newcommand{\ZZ}{\mathbb{Z}}
\newcommand{\RR}{\mathbb{R}}
\newcommand{\QQ}{\mathbb{Q}}
\newcommand{\ww}{\mathsf{w}}
\newcommand{\ff}{\mathsf{f}}
\newcommand{\lnor}{\mathbin{\overline{\lor}}}
\newcommand{\lnand}{\mathbin{\overline{\land}}}
\newcommand{\universe}{\mathcal{U}}
\newcommand{\powerset}[1]{\mathcal{P}(#1)}
\newcommand{\definedby}{\mathbin{:=}}

% For proofs:
\renewcommand{\qedsymbol}{\ensuremath{\blacksquare}}

% To insert a new page:
\newcommand{\blankpage}{\newpage\mbox{}}

% For inline diagrams
\newcommand{\InlineDiag}[1]{\ensuremath{\begin{tikzcd}#1\end{tikzcd}}}

% Basic Category Theory
% Tom Leinster <Tom.Leinster@ed.ac.uk>
%
% Copyright (c) Tom Leinster 2014-2016
%
% Macros
%


% PACKAGES


\usepackage{latexsym}
\usepackage{amssymb,amsmath}
\usepackage{bbm}
\usepackage{mathrsfs}

\usepackage[all,2cell]{xy}
\UseAllTwocells
\xyoption{arc}
\xyoption{rotate}

% STYLE PARAMETERS AND HYPHENATION


\def\makeRRlabeldot#1{\hss\llap{#1}}    % Redefines macro in cambridge7A.cls
\renewcommand{\theenumi}{\rm(\alph{enumi})}
\renewcommand{\theequation}{\thechapter.\arabic{equation}}

\hyphenation{mo-noi-dal mo-noid co-do-main pre-sh-eaf pre-sh-ea-ves}


% FORMATTING AND ENGLISH TYPOGRAPHIC


\newcommand{\demph}[1]{\textbf{#1}}     % Emphasis used for terms being
                                        % defined

\newenvironment{displaytext}%   % Used for several text-display environments
{\begin{list}{}%
{\setlength{\leftmargin}{2em}%
\setlength{\rightmargin}{\leftmargin}}%
\item}%
{\end{list}}

\newenvironment{setprop}{\begin{displaytext}\itshape}{\end{displaytext}}
% Properties in set theory chapter

\newenvironment{slogan}{\begin{displaytext}\itshape}{\end{displaytext}}

\newenvironment{citedsource}{\begin{displaytext}}{\end{displaytext}}
% Cited sources in Further Reading chapter

\newcommand{\minihead}[1]{\subsection*{#1}}
% Unnumbered heading in text

\newcommand{\subjectchange}{\bigskip}
% Small vertical space in text for change of subject

\newcommand{\exs}{\minihead{Exercises}}
\newcommand{\exone}{\minihead{Exercise}}


% MATHEMATICAL TYPOGRAPHIC


\newcommand{\cat}[1]{\mathscr{#1}}      % Arbitrary category
\newcommand{\scat}[1]{\mathbf{#1}}      % Arbitrary small category
\newcommand{\fcat}[1]{\mathbf{#1}}      % Typeface for fixed categories

\newcommand{\ovln}[1]{\overline{#1}}
\newcommand{\lwr}[1]{\mathbf{#1}}       % \lwr{n} denotes n-element set
\newcommand{\rreg}[1]{\underline{#1}}   % Regular representation
\newcommand{\ynt}[1]{\tilde{#1}}        % Nat transf in Yoneda lemma
\newcommand{\yel}[1]{\hat{#1}}          % Element in Yoneda lemma
\newcommand{\wideyel}[1]{\widehat{#1}}  % Wide version of \yel


% CONSTANTS (LETTERS)


\newcommand{\epsln}{\varepsilon}        % \epsilon is not used anywhere

\newcommand{\nat}{\mathbb{N}}           % Natural numbers
\newcommand{\integers}{\mathbb{Z}}
\newcommand{\rationals}{\mathbb{Q}}
\newcommand{\reals}{\mathbb{R}}
\newcommand{\complexes}{\mathbb{C}}

\DeclareMathOperator{\cset}{\cat{C}}    % Set of closed subsets
\DeclareMathOperator{\Cl}{Cl}           % Closure
\DeclareMathOperator{\Cone}{Cone}       % Cones (in limit sense)
\DeclareMathOperator{\End}{\fcat{End}}  % Endomorphisms
\DeclareMathOperator{\Eq}{Eq}           % Equalizer
\DeclareMathOperator{\ev}{ev}           % Evaluation
\DeclareMathOperator{\Hom}{Hom}         % Hom-set/functor
\DeclareMathOperator{\HOM}{\fcat{Hom}}  % Homs between vector spaces
\DeclareMathOperator{\im}{im}           % Image
\DeclareMathOperator{\Lan}{Lan}         % Left Kan extension
\DeclareMathOperator{\ob}{ob}           % Object-set
\DeclareMathOperator{\oset}{\cat{O}}    % Set of open subsets
\DeclareMathOperator{\pr}{pr}           % Projection
\DeclareMathOperator{\pset}{\cat{P}}    % Power set
\DeclareMathOperator{\Sub}{Sub}         % Set of subobjects

\newcommand{\h}{H}                      % Homs (e.g. \h_A means Hom(-, A))
\newcommand{\op}{\mathrm{op}}           % Opposite category
\newcommand{\true}{\texttt{true}}       % Truth value "true"
\newcommand{\false}{\texttt{false}}     % Truth value "false"

\newcommand{\One}{\mathbf{1}}           % Terminal category
\newcommand{\Two}{\mathbf{2}}           % Single-arrow category
\newcommand{\Ab}{\fcat{Ab}}             % Abelian groups
\newcommand{\Bilin}{\fcat{Bilin}}       % Bilinear maps
\renewcommand{\Cat}{\fcat{Cat}}           % Small categories
\newcommand{\CAT}{\fcat{CAT}}           % All categories
\newcommand{\CptHff}{\fcat{CptHff}}     % Compact Hausdorff spaces
\newcommand{\CRing}{\fcat{CRing}}       % Commutative rings
\newcommand{\FDVect}{\fcat{FDVect}}     % Fin-dim vector spaces
\newcommand{\Field}{\fcat{Field}}       % Fields
\newcommand{\FinSet}{\fcat{FinSet}}     % Finite sets
\newcommand{\Fix}{\fcat{Fix}}           % Fixed-point category
\newcommand{\Grp}{\fcat{Grp}}           % Groups
\newcommand{\Mt}{\fcat{Mat}}            % Matrices
\newcommand{\Mon}{\fcat{Mon}}           % Monoids
\newcommand{\Ord}{\fcat{Ord}}           % Total orders
\newcommand{\Ring}{\fcat{Ring}}         % Rings
\newcommand{\Set}{\fcat{Set}}           % Sets
\newcommand{\Sym}{\fcat{Sym}}           % Permutations
\newcommand{\Tp}{\fcat{Top}}            % Topological spaces
\newcommand{\Toph}{\fcat{Toph}}         % Topo spaces + hty classes of maps
\newcommand{\Vect}{\fcat{Vect}}         % Vector spaces


% CONSTANT SYMBOLS


\newcommand{\such}{\mathrel{|}}         % Separator in set-builder notation
\newcommand{\Bigsuch}{\mathrel{\big|}}  % Big version of \such
\newcommand{\without}{\setminus}        % Set complement

\newcommand{\emptybk}{\hspace*{.5em}}   % Empty space used as blank
\newcommand{\blank}{(\emptybk)}         % Generic blank
\newcommand{\dashbk}{-}                 % Dash used as blank
\newcommand{\bl}{\bullet}               % Bullet used as blank

\newcommand{\from}{{\colon}\linebreak[0]}       % Colon for f: A --> B

\newcommand{\iso}{\cong}                % Isomorphism
\newcommand{\eqv}{\simeq}               % Equivalence
\newcommand{\sub}{\subseteq}            % Subset (possibly not proper)
\newcommand{\ladj}{\dashv}              % Left adjoint
\newcommand{\divides}{\mathbin{\mid}}   % For divisibility of integers
\newcommand{\of}{\mathbin{\circ}}       % Composition

\newcommand{\meet}{\wedge}              % Joins and meets in ordered sets
\newcommand{\Meet}{\bigwedge}
\newcommand{\join}{\vee}
\renewcommand{\Join}{\bigvee}


% MATHEMATICAL EXPRESSIONS TAKING ARGUMENTS


\newcommand{\ftrcat}[2]{[#1,#2]}                % Functor category
\newcommand{\pshf}[1]{\ftrcat{#1^\op}{\Set}}    % Presheaf category
\newcommand{\psh}[1]{\hat{#1}}                  % Alternative pshf cat

\newcommand{\elt}[1]{\scat{E}(#1)}              % Category of elements

\newcommand{\colt}[1]{\lim\limits_{\rightarrow #1}}     % Colimit
\newcommand{\lt}[1]{\lim\limits_{\leftarrow #1}}        % Limit

\newcommand{\abel}[1]{#1_{\mathrm{ab}}}                 % Abelianization
\newcommand{\comma}[2]{(#1 \mathbin{\Rightarrow} #2)}   % Comma cat
\newcommand{\qer}[2]{#1/\mathord{#2}}   % Eq reln for use after slash
\newcommand{\crd}[1]{\left|#1\right|}   % Cardinality


% ARROWS


\newcommand{\oppairu}{\rightleftarrows}         % Unlabelled opposing pair
\newcommand{\oppairua}{\leftrightarrows}        % Same, but reversed
\newcommand{\parpairu}{\rightrightarrows}       % Unlabelled parallel pair
\newcommand{\ot}{\leftarrow}                    % Left-pointing arrow
\newcommand{\toby}[1]{\stackrel{#1}{\longrightarrow}}   % Labelled arrow
\newcommand{\otby}[1]{\stackrel{#1}{\longleftarrow}}    % Left lblld arrow
\newcommand{\longto}{\longrightarrow}           % Long labelled arrow
\newcommand{\toiso}{\toby{\textstyle{}_\sim}}   % Isomorphism arrow
\newcommand{\incl}{\hookrightarrow}             % Inclusion arrow
\renewcommand{\mapsfrom}{\mathrel{\reflectbox{\ensuremath{\mapsto}}}}
% Left-pointing \mapsto
\newcommand{\textiff}{\Longleftrightarrow}      % Iff arrow for use in text
\newcommand{\textif}{\Longleftarrow}            % If arrow for use in text
\newcommand{\textonlyif}{\Longrightarrow}       % Only if arrow for text


\newcommand{\nent}{\rotatebox{45}{$\Rightarrow$}}
% North-east-pointing arrow for natural transformations
\newcommand{\nwnt}{\rotatebox{-45}{$\Leftarrow$}}       % North-west
\newcommand{\swnt}{\rotatebox{45}{$\Leftarrow$}}        % South-west
\newcommand{\sent}{\rotatebox{-45}{$\Rightarrow$}}      % South-east
\newcommand{\neeq}{\rotatebox{45}{$=$}}
% North-east pointing equality

\newcommand{\oppair}[4]{%
\xymatrix{%
#1 \ar@<.5ex>[r]^-{#3} &{#2}\ar@<.5ex>[l]^-{#4}%
}}
% Opposite-pointing pair of arrows, with arguments, for use in displays

\newcommand{\oppairi}[4]{%
\xymatrix@1{%
#1 \ar@<.5ex>[r]^{#3} &{#2}\ar@<.5ex>[l]^{#4}%
}}
% Opposite-pointing pair of arrows, with arguments, for use in text

\newcommand{\parpair}[4]{%
\xymatrix{%
#1 \ar@<.5ex>[r]^{#3} \ar@<-.5ex>[r]_{#4} &#2%
}}
% Parallel pair of arrows, with arguments, for use in displays

\newcommand{\parpairi}[4]{%
\xymatrix@1{%
#1 \ar@<.5ex>[r]^{#3} \ar@<-.5ex>[r]_{#4} &#2%
}}
% Parallel pair of arrows, with arguments, for use in text

\newcommand{\adjn}[4]{%
\xymatrix{
#1 \ar@{}[d]|\ladj \ar@<1ex>[d]^{#4} \\
#2 \ar@<1ex>[u]^{#3}
}}
% Adjunction, with arguments, for use in displays

\newcommand{\hadjnli}[4]{%
\xymatrix@1{
#1 \ar@<1.1ex>[r]^-{#3} \ar@{}[r]|-\bot &#2 \ar@<1.1ex>[l]^-{#4}}}
% Adjunction, with arguments, for use in text (left adjt on top)

\newcommand{\hadjnri}[4]{%
\xymatrix@1{
#1 \ar@<1.1ex>[r]^-{#3} \ar@{}[r]|-\top &#2 \ar@<1.1ex>[l]^-{#4}}}
% Adjunction, with arguments, for use in text (left adjt on bottom)

\newcommand{\searrows}{\makebox[1em]{$\downarrow$%
\hspace*{-.2em}\raisebox{-1ex}{$\rightarrow$}}}
% Little diagram of anticlockwise path around a commutative square

\newcommand{\esarrows}{\makebox[1em]{%
\raisebox{1ex}{$\rightarrow$}\hspace*{-.2em}$\downarrow$}}
% Little diagram of clockwise path around a commutative square


% PICTURE ENVIRONMENT


\newcommand{\cell}[4]{\put(#1,#2){\makebox(0,0)[#3]{\ensuremath{#4}}}}
% Places a label in picture environment


% REFERENCES, CITATIONS AND INDEXING


\newcommand{\bref}[1]{\ref{#1}}         % \ref in brackets

\newcommand{\citestyle}[1]{#1}  % Style for sources cited
\newcommand{\citeGow}{\citestyle{Gowers (2002)}}
\newcommand{\citeKel}{\citestyle{Kelly (1982)}}
\newcommand{\citeLR}{\citestyle{Lawvere and Rosebrugh (2003)}}
\newcommand{\citeRST}{\citestyle{Leinster (2014)}}
\newcommand{\citeCWM}{\citestyle{Mac~Lane (1971)}}

% The following commands are for the index of notation

\newcommand{\ntn}[1]{\label{notn:#1}}   % Make label
\newcommand{\nref}[1]{\pageref{notn:#1}}% Refers to page where notn is used
\newcommand{\nuse}[2]{#2, #1\\}
% First arg is page ref; second is description of notation
